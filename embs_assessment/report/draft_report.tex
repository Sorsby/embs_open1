%%%%%%%%%%%%%%%%%%%%%%%%%%%%%%%%%%%%%%%%%
% Simple Sectioned Essay Template
% LaTeX Template
%
% This template has been downloaded from:
% http://www.latextemplates.com
%
% Note:
% The \lipsum[#] commands throughout this template generate dummy text
% to fill the template out. These commands should all be removed when 
% writing essay content.
%
%%%%%%%%%%%%%%%%%%%%%%%%%%%%%%%%%%%%%%%%%

%----------------------------------------------------------------------------------------
%	PACKAGES AND OTHER DOCUMENT CONFIGURATIONS
%----------------------------------------------------------------------------------------

\documentclass[12pt]{article} % Default font size is 12pt, it can be changed here

\usepackage{geometry} % Required to change the page size to A4
\geometry{a4paper} % Set the page size to be A4 as opposed to the default US Letter

\usepackage{graphicx} % Required for including pictures

\usepackage{float} % Allows putting an [H] in \begin{figure} to specify the exact location of the figure
\usepackage{wrapfig} % Allows in-line images such as the example fish picture

\usepackage{amsmath}

\linespread{1.2} % Line spacing

%\setlength\parindent{0pt} % Uncomment to remove all indentation from paragraphs

\graphicspath{{Pictures/}} % Specifies the directory where pictures are stored

\begin{document}

%----------------------------------------------------------------------------------------
%	TITLE PAGE
%----------------------------------------------------------------------------------------

\begin{titlepage}

\newcommand{\HRule}{\rule{\linewidth}{0.5mm}} % Defines a new command for the horizontal lines, change thickness here

\center % Center everything on the page

\textsc{\LARGE University of York}\\[1.5cm] % Name of your university/college
\textsc{\Large Embedded Systems Design \& Implementation}\\[0.5cm] % Major heading such as course name
\textsc{\large Open Individual Assessment}\\[0.5cm] % Minor heading such as course title

\HRule \\[0.4cm]
{ \huge \bfseries Open Assessment 1}\\[0.4cm] % Title of your document
\HRule \\[1.5cm]

\begin{minipage}{0.4\textwidth}
\begin{flushleft} \large
\center \emph{Examination number:}\\
\center Y3606797
\end{flushleft}
\end{minipage}

%\includegraphics{Logo}\\[1cm] % Include a department/university logo - this will require the graphicx package

\vfill % Fill the rest of the page with whitespace

\end{titlepage}

%----------------------------------------------------------------------------------------
%	TABLE OF CONTENTS
%----------------------------------------------------------------------------------------

\tableofcontents % Include a table of contents

\newpage % Begins the essay on a new page instead of on the same page as the table of contents 

%----------------------------------------------------------------------------------------
%	INTRODUCTION
%----------------------------------------------------------------------------------------

\section{Part 1 - Theory} % Major section

%------------------------------------------------

\subsection{Question 1} % Sub-section

\paragraph{}
We can determine the rate X of actor H by producing a set of simultaneous equations from Table 1 and the provided Synchronous Dataflow model.

The topology matrix for the SDF model is as follows:
\\

$
\Gamma = \begin{bmatrix}
		2 & 0 & 0 & 0 & -2 & 0 & 0 & 0 & 0 \cr
		0 & 2 & 0 & 0 & -2 & 0 & 0 & 0 & 0 \cr
		0 & 0 & 2 & 0 & -2 & 0 & 0 & 0 & 0 \cr
		0 & 0 & 0 & 2 & -2 & 0 & 0 & 0 & 0 \cr
		0 & 0 & 0 & 0 & 2 & -6 & 0 & 0 & 0 \cr
		0 & 0 & 0 & 0 & 0 & 1 & 0 & 0 & -1 \cr
		0 & 0 & 0 & 0 & 0 & 0 & 3 & -2 & 0 \cr
		0 & 0 & 0 & 0 & 0 & 0 & 0 & X & -3 \cr
		0 & 0 & 0 & 0 & 0 & 0 & -1 & 0 & 1 \cr
		\end{bmatrix}
$
\\ \\
This gives us the following simultaneous equations:
\\ \\
$
\begin{matrix}
2A - 2E = 0 & 2B - 2E = 0 & 2C - 2E = 0 \\
2D - 2E = 0 & 2E - 6F = 0 & F - I = 0 \\
3G - 2H = 0 & XH - 3I = 0 & I - G = 0
\end{matrix}
$
\\

\begin{minipage}{0.6\textwidth}
\begin{flushleft}
Using these equations I determined that $X = 2$. Similarly, I determined the firing frequencies of the remaining actors, seen in the vector q:
\end{flushleft}
\end{minipage}
~
\begin{minipage}{0.2\textwidth}
\begin{flushright}
$
q = \bordermatrix{~ & ~ \cr
				A & 6 \cr
				B & 6 \cr
				C & 6 \cr
				D & 6 \cr
				E & 6 \cr
				F & 2 \cr
				G & 2 \cr
				H & 3 \cr
				I & 2 \cr}	
$
\end{flushright}
\end{minipage}\\

%------------------------------------------------

\subsection{Question 2} % Sub-section
Using the firing frequencies determined in Question 1, I was able to identify the following PASS schedule: \\

a.fire(3); b.fire(3); c.fire(3); d.fire(3); e.fire(3); f.fire(1); a.fire(3); b.fire(3); c.fire(3); d.fire(3); e.fire(3); f.fire(1); g.fire(2); h.fire(3); i.fire(2);
\\

The maximum required FIFO buffer size is 6 as required and the number of firings of the actors match up with their frequencies in the vector q (Question 1).

%------------------------------------------------

\subsection{Question 3} % Sub-section
For my chosen PASS schedule the number of tokens that must be initially stored in the buffer of the feedback channel c9 is \large 2.


%------------------------------------------------

%----------------------------------------------------------------------------------------
%	MAJOR SECTION 1
%----------------------------------------------------------------------------------------

\section{Part 2 - WSN MAC layer protocol} % Major section


%------------------------------------------------

\subsection{Question 1} % Sub-section
For this task I opted to dedicate an entire class (SourceNodeActor) to perform the PtolemyII actor functions needed in the simulation. The class SourceNodeActor relies on another class (SourceNode) which provides the implementation of the protocol features, this in turn relies on another class SinkNodeModel which is responsible for modelling the sink nodes parameters and ultimately synchronising with their reception phases.
\\
An instance of SourceNode is created with a number of channels provided to the constructor (SourceNodeActor, initialise, 62). This design was taken so that I could reuse the code during the second part of this question.
\\
When the actor is initialised it begins reading beacons from a sink node channel for a specified amount of time, given by the constant DEFAULT_LISTENING_TIME (SourceNode, 11), when we read a beacon we extend the time to try get another and thus calculate the sink parameters. If the time expires we switch nodes and do the same again - this is how I sync to multiple nodes effectively.
\\
When n=1 we have to wait an entire protocol length to read another beacon, so we can switch away and come back later - we then use the length between the iterations to calculate the sink parameters.

%------------------------------------------------

\subsection{Question 2} % Sub-sub-section

%------------------------------------------------

%----------------------------------------------------------------------------------------
%	MAJOR SECTION X - TEMPLATE - UNCOMMENT AND FILL IN
%----------------------------------------------------------------------------------------

\section{Part 3 - Embedded platform modelling}

\subsection{Question 1}

\end{document}